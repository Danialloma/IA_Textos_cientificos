\documentclass[conference]{IEEEtran}

\IEEEoverridecommandlockouts
% The preceding line is only needed to identify funding in the first footnote. If that is unneeded, please comment it out.
\usepackage{cite}
\usepackage{amsmath,amssymb,amsfonts}
\usepackage{algorithmic}
\usepackage{graphicx}
\usepackage{textcomp}
\usepackage{xcolor}
\usepackage{hyperref}
\usepackage{microtype}
\usepackage[english, spanish]{babel}
\usepackage[T1]{fontenc}
\usepackage[utf8]{inputenc}
\def\BibTeX{{\rm B\kern-.05em{\sc i\kern-.025em b}\kern-.08em
    T\kern-.1667em\lower.7ex\hbox{E}\kern-.125emX}}
\begin{document}

\title{Conference Paper Title*\\
{\footnotesize \textsuperscript{*}Note: Sub-titles are not captured in Xplore and
should not be used}
\thanks{}
}

\author{\IEEEauthorblockN{1\textsuperscript{st} Luis Alan Morales Castillo}
\IEEEauthorblockA{\textit{Departamento de ingenieria y ciencias} \\
\textit{Tecnológico de Monterrey}\\
Ciudad de México, México\\
a01659147@tec.mx}
\and
\IEEEauthorblockN{2\textsuperscript{nd} Daniel Alejandro López Martínez}
\IEEEauthorblockA{\textit{Departamento de ingenieria y ciencias} \\
\textit{Tecnológico de Monterrey}\\
Ciudad de México, México \\
a01770442@tec.mx}
\and
\IEEEauthorblockN{3\textsuperscript{rd} Given Name Surname}
\IEEEauthorblockA{\textit{dept. name of organization (of Aff.)} \\
\textit{name of organization (of Aff.)}\\
City, Country \\
email address or ORCID}
\and
\IEEEauthorblockN{4\textsuperscript{th} Given Name Surname}
\IEEEauthorblockA{\textit{dept. name of organization (of Aff.)} \\
\textit{name of organization (of Aff.)}\\
City, Country \\
email address or ORCID}
\and
\IEEEauthorblockN{5\textsuperscript{th} Given Name Surname}
\IEEEauthorblockA{\textit{dept. name of organization (of Aff.)} \\
\textit{name of organization (of Aff.)}\\
City, Country \\
email address or ORCID}
\and
\IEEEauthorblockN{6\textsuperscript{th} Given Name Surname}
\IEEEauthorblockA{\textit{dept. name of organization (of Aff.)} \\
\textit{name of organization (of Aff.)}\\
City, Country \\
email address or ORCID}
}

\maketitle

\begin{abstract}
This document is a model and instructions for \LaTeX.
This and the IEEEtran.cls file define the components of your paper [title, text, heads, etc.]. *CRITICAL: Do Not Use Symbols, Special Characters, Footnotes, 
or Math in Paper Title or Abstract.
\end{abstract}

\begin{IEEEkeywords}
component, formatting, style, styling, insert
\end{IEEEkeywords}

\section{Introduction}
This document is a model and instructions for \LaTeX.
Please observe the conference page limits. 

\section{Criptografia}

Actualmente, blockchain ha ganado un amplio reconocimiento en investigación y una
gran atención pública en el campo de la innovación a nivel mundial (Lu Yang, 2019).

En la vida real confiamos en los bancos y en un banco central para llevar un registro, 
en lugar de eso todos tenemos acceso a este registro: Blockchain como dice su nombre es 
una cadena de bloques en un libro de contabilidad digital, las páginas de este libro están 
conectadas matemáticamente por lo tanto sólo puedes agregar una nueva hoja a la cadena 
resolviendo un problema matemático. La tecnología blockchain, gracias a su naturaleza de 
descentralización y seguridad, tiene un gran potencial para mejorar los servicios y aplicaciones de big data (N.Deepa 2022). 

Hoy el uso más comun es en el mundo de las criptomonedas como Bitcoin y sirve para tener de forma clara y permanente 
todas las transacciones. En las criptomonedas cada una tiene un tipo diferente de mecanismo de 
consenso como Proof of Work, Proof of Stake, Ouroboros, etc. Estos involucran a los mineros, 
la resolución del problema matemático que implementa técnicas de criptografía que usan Hashes.

\section{Vulnerabilidades en web y nube}
El uso de Pruebas de Conocimiento Cero  surge ante la necesidad de autenticar usuarios y verificar transacciones sin revelar información sensible (Zhang et al., 2024; Berrios Moya et al., 2025). Estas tecnologías buscan resolver fallos de autenticación y control de acceso que comprometen la seguridad en aplicaciones web y herramientas en la nube, como los identificados por OWASP top 10 (Fredj et al., 2021).
En la nube, la Gestión de Identidad y Acceso (IAM) sigue siendo una fuente crítica de vulnerabilidades por errores en las configuraciones (Tabrizchi & Rafsanjani, 2020). Los atacantes aprovechan estas fallas mediante técnicas como DDoS o escalamiento de privilegios (Berrios Moya et al., 2025). La mitigación requiere un enfoque integral que combine la seguridad a nivel de aplicación (validación contra SQL Injection y XSS) con modelos criptográficos avanzados.
Las arquitecturas emergentes proponen blockchain y ZKP para descentralizar la confianza: los contratos inteligentes gestionan la lógica de acceso y las pruebas criptográficas permiten verificar credenciales sin exponer datos (Zhang et al., 2024; Berrios Moya et al., 2025).
Aun con tecnologías ZKP y blockchain, las amenazas como la manipulación o el fallo en control de acceso persisten (Berrios Moya et al., 2025). OWASP las clasifica entre las vulnerabilidades más críticas (Fredj et al., 2021). En la nube, los errores de IAM aumentan el riesgo (Tabrizchi & Rafsanjani, 2020), mientras que en blockchain surgen vulnerabilidades en smart contracts (Zhang et al., 2024). Esto evidencia que la lógica de autorización sigue siendo el eslabón débil, y que las ZKP pueden reforzarla mediante verificación criptográfica sin exposición de datos.
La autenticación y el control de acceso continúan siendo los principales vectores de riesgo en web y nube. Las Pruebas de Conocimiento Cero y el blockchain surgen como estrategias prometedoras para reducir estos riesgos, fortaleciendo la privacidad y la autorización (Berrios Moya et al., 2025; Zhang et al., 2024). Sin embargo, la evidencia actual se basa en modelos teóricos más que experimentales, lo que plantea la necesidad de evaluar cómo los escáneres de vulnerabilidades tradicionales enfrentan los nuevos escenarios de seguridad (Fredj et al., 2021).


\section*{Referencias}

\begin{itemize}
    \item 
    \href{https://doi.org/10.1016/j.jii.2019.04.002}{The blockchain: State-of-the-art and research challenges (Journal Of Industrial Information Integration, Lu, Y., 2019)}

    \item 
    \href{https://doi.org/10.1016/j.future.2022.01.017}{A survey on blockchain for big data: Approaches, opportunities, and future directions (Future Generation Computer Systems, Deepa, N., et al., 2022)}

    \item \begin{otherlanguage}{english}
    \href{https://doi.org/10.3390/s25113450}{A Zero-Knowledge Proof-Enabled Blockchain-Based Academic Record Verification System (Sensors, Berrios Moya, J. A., Ayoade, J., \& Uddin, M. A., 2025)}
    \end{otherlanguage}
    \item 
    \href{https://doi.org/10.1007/978-3-030-68887-5_14}{An OWASP Top Ten Driven Survey on Web Application Protection Methods (Springer International Publishing, Fredj, O. B., et al., 2021)}

    \item 
    \href{https://doi.org/10.1007/s11227-020-03213-1}{A survey on security challenges in cloud computing: Issues, threats, and solutions (Journal of Supercomputing, Tabrizchi, H., \& Kuchaki Rafsanjani, M., 2020)}

    \item 
    \href{https://doi.org/10.3390/electronics13214260}{A Blockchain and Zero Knowledge Proof Based Data Security Transaction Method in Distributed Computing (Electronics, Zhang, B., et al., 2024)}
\end{itemize}

\end{document}
